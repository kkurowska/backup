\documentclass[twoside,a4paper,12pt]{mwbk}

\usepackage[utf8]{inputenc}
%\usepackage{times}
\usepackage{amsmath,amssymb,amsthm}
\usepackage{xcolor}
\usepackage[final]{pdfpages}
\usepackage{graphicx}
%\usepackage[nottoc]{tocbibind}
\usepackage{caption}
\usepackage{subcaption}
\captionsetup{compatibility=false}
\usepackage{dsfont}
\usepackage{float}

\usepackage{booktabs}
\usepackage{array}

\renewcommand{\labelitemi}{$\bullet$}
\renewcommand{\labelitemii}{$\cdot$}
\renewcommand{\labelitemiii}{$\diamond$}
\renewcommand{\labelitemiv}{$\ast$}


\setlength{\parindent}{0pt}

%\usepackage{geometry}
%\newgeometry{tmargin=2.5cm, bmargin=2.5cm, lmargin=2.5cm, rmargin=2.5cm}

%\numberwithin{equation}{section}
%\numberwithin{figure}{section}
\renewcommand{\thefigure}{\thechapter.\arabic{figure}}

%\newcommand*{\doi}[1]{\href{http://dx.doi.org/#1}{doi: #1}}
%\newcommand*{\MR}[1]{\href{http://www.ams.org/mathscinet-getitem?mr=#1&return=pdf}{MR #1}}
%\newcommand*{\ZBL}[1]{\href{http://www.zentralblatt-math.org/zmath/en/advanced/?q=an:#1&format=complete}{Zbl #1}}


\newcommand{\1}[1]{\mathds{1}\left(#1\right)}

%\newenvironment{diagrams}[2]{\begin{figure}[p]
%	%\centering
%	\begin{subfigure}{.5\textwidth}
%		\centering
%		\includegraphics[width=\textwidth]{wykresy/#1.png}
%		\caption{}
%		\label{#1}
%	\end{subfigure}
%	\begin{subfigure}{.5\textwidth}
%		\centering
%		\includegraphics[width=\textwidth]{wykresy/#2.png}
%		\caption{}
%		\label{#2}
%	\end{subfigure}
%	}
%	{
%	\end{figure}
%	}

\newenvironment{mainsubdiagrams2}[4]{
	\begin{subfigure}{.45\textwidth}
		\centering
		\includegraphics[width=.9\textwidth]{graphs/#1.png}
		\label{fig:#1}
		\caption{#2}
	\end{subfigure}
	\begin{subfigure}{.45\textwidth}
		\centering
		\includegraphics[width=.9\textwidth]{graphs/#3.png}
		\label{fig:#3}
		\caption{#4}
	\end{subfigure}
} {}

\newenvironment{subdiagrams2}[4]{
		\begin{subfigure}{.45\textwidth}
			\centering
			\includegraphics[width=.5\textwidth]{graphs/#1.png}
			\label{fig:#1}
			\caption{#2}
		\end{subfigure}
		\begin{subfigure}{.45\textwidth}
			\centering
			\includegraphics[width=.5\textwidth]{graphs/#3.png}
			\label{fig:#3}
			\caption{#4}
		\end{subfigure}
	} {}

\newenvironment{subdiagrams3}[6]{
	\begin{subfigure}{.3\textwidth}
		\centering
		\includegraphics[width=.8\textwidth]{graphs/#1.png}
		\label{fig:#1}
		\caption{#2}
	\end{subfigure}
	\begin{subfigure}{.3\textwidth}
		\centering
		\includegraphics[width=.8\textwidth]{graphs/#3.png}
		\label{fig:#3}
		\caption{#4}
	\end{subfigure}
	\begin{subfigure}{.3\textwidth}
	\centering
	\includegraphics[width=.8\textwidth]{graphs/#5.png}
	\label{fig:#5}
	\caption{#6}
	\end{subfigure}
} {}

\newenvironment{subdiagram}[2]{
	\begin{subfigure}{\textwidth}
		\centering
		\includegraphics[width=.65\textwidth]{graphs/#1.png}
		\label{#1}
		\caption{#2}
	\end{subfigure}
} {}

\theoremstyle{plain}
\newtheorem{thm}{Theorem}[chapter]

\theoremstyle{definition}
\newtheorem{defn}[thm]{Definition}


\begin{document}
	
\begin{titlepage}
	\includepdf{title_page.pdf}
\end{titlepage}


\tableofcontents

\chapter*{Introduction}
Few words about my thesis...

Generally what is a data mining, what are the wavelets, what is the main purpose - edge detection.
\chapter{Wavelets theory}
A ''wavelet'' literally means a small wave. This term says a lot about its nature. Wavelets are a family of functions which oscillates like wave and should be compactly supported. Additionally, the wavelet has zero mean.

\begin{defn}
Wavelets are created by scaling and shifting of the, so called, mother wavelet $\psi(t)$. The child wavelets are defined as

\begin{equation}
\label{eq:wavelets}
\psi^{(a,b)}(t)=|a|^{-\frac{1}{2}} \psi\left(\frac{t-b}{a}\right),\ a>0,
\end{equation}

where $a$ is a scale parameter and $b$ translation parameter.

\end{defn}

There are plenty of different mother wavelets, for example

\begin{figure}[h]
	\centering
	\includegraphics[width=\textwidth]{wavelets_with_bottom_line.png}
	\caption{Different types of wavelets.}
	\label{fig:wavelets}
\end{figure}

\textit{** I'll add more about wavelets properties - orthogonal basis, decorrelated coefficients, vanishing moments, compact support. And how those properties can be useful in data mining. **}


\section{Haar and Daubechies wavelets}

\textit{** Add that db1 is the same as Haar. Add little more about specific properties for this wavelet family. **}

Each type of wavelet function is more suitable for different applications. The best for image analysis are the Daubechies wavelets. 

\begin{defn}
Daubechies wavelets are collection of orthogonal and compactly supported functions. A denotation for those wavelets is $dbN$, where $N$ means a maximal number of vanishing moments.
\end{defn}

\begin{figure}[h]
	\centering
	\includegraphics[width=\textwidth]{DB_N.png}
	\caption{Daubechies wavelets.}
	\label{fig:db_wavelets}
\end{figure}

\section{Wavelet Transform}

Generally, integral transforms are useful in data processing. Their idea is to convert data into another domain, where it is easier to manipulate them, finding desired informations, etc. Finally, the results are transformed back to the original domain by inverse integral transform.

We can define integral transform based on the wavelet functions - Wavelet Transform.
However, the most popular kind of integral transform is Fourier Transform. Here the question arises, what are the differences between both operators. Lets start from the definitions.

\begin{defn}
Continuous Wavelet Transform is expressed by the formula

\begin{equation}
W(a,b)=\int_{-\infty}^{\infty} y(t) a^{-\frac{1}{2}} \psi\left(\frac{t-b}{a}\right) dt,
\end{equation}

where $a$ is scale parameter, $b$ translation parameter and $y(t)$ original signal.
\end{defn}


\begin{defn}
Fourier transform is defined as

\begin{equation}
Y(f)=\int_{-\infty}^{\infty} y(t) e^{-i\omega t} dt,
\end{equation}

where $y(t)$ is time domain signal and $Y(f)$ is frequency domain signal.
\end{defn}

The most important differences are presented in the table below.

\begin{table}[h]
\centering
\begin{tabular}{|p{0.5\linewidth}|p{0.5\linewidth}|}
\toprule
\textbf{ Wavelet transform} & \textbf{Fourier transform}
\\ \midrule
Suitable for stationary and non-\allowbreak -stationary signals 
& Suitable for stationary signals 
\\ \midrule
High time and frequency resolution
& Zero time resolution and very high frequency resolution     
\\ \midrule
Very suitable for studying the local behaviours of the signal
& No suitable  
\\ \midrule
Scaled and translated mother wavelets
& Sine and cosine waves
\\ \bottomrule
\end{tabular}
%\label{tab:wt_vs_ft}
\caption{The comparison of Wavelet and Fourier Transform.}
\end{table}

The obvious distinction between both transforms is the type of function. In Fourier case there are sine and cosine functions, wherein wavelet transform uses wavelets. Sine function oscillates on the whole real axis, thus it cannot represent abrupt changes. However, the Wavelet transform is localized in space and time, so it can be used to detect trends or sudden changes in signals and images. Moreover, wide range of wavelet functions is a main advantage of wavelet analysis, because we can adjust the type of wavelet to the specific case and thank to that obtain more accurate results.

\section{Discrete Wavelet transform}

There are two types of Wavelet Transform:
\begin{itemize}
	\item Continuous Wavelet Transform (CWT),
	\item Discrete Wavelet Transform (DWT).
\end{itemize}

The Discrete Wavelet Transform has a wide range of applications in image processing, for example denoising and compressing, also in detecting shapes on images.

\textit{** I'll add a little more about applications in pre-processing and machine learning **}

Scale and translation parameters are defined as

\begin{equation}
a = 2^j \text{ and } b = 2^j m,\ j,m=1,2,\ldots.
\end{equation}

Therefore, assuming that $0 \leq t < 2^K$, $K \in \mathbb{Z}$, the coefficients of Discrete Wavelet Transform are defined as follow

\begin{equation}
d^{(j,m)}=\sum_{t=0}^{2^K -1} y(t) 2^{-\frac{j}{2}} \psi\left(2^{-j}t - m\right).
\end{equation}

The figure \ref{fig:DWT} on a page \pageref{fig:DWT} shows how DWT works. Discrete Wavelet Transform splits signal with two filters: $g(n)$ - low pass filter (LPF) and $h(n)$ - high pass filter (HPF). The LPF captures a part with lower frequencies which refers to the main signal. Whereas, the HPF captures higher frequencies - a noise of the signal. Subsequently, both parts are downsampled by a factor of 2. This decomposition can be repeated on the LPF part of the signal. Hence, the next levels of DWT coefficients.

\begin{figure}[h]
	\centering
	\includegraphics[width=\textwidth]{DWT.png}
	\caption{Discrete Wavelet transform on a signal $x(n)$.}
	\label{fig:DWT}
\end{figure}


\section{2-D Discrete Wavelet transform}
\label{sec:2D_DWT}

2-Dimensional Discrete Wavelet Transform works similar way as 1-D with High Pass Filter, Low Pass Filter and downsampling, except that one level of the decomposition includes double filtering, on columns and rows. The figure \ref{fig:2D_DWT} shows an image decomposition. Firstly, the DWT is applied on columns of the input image and then on the rows of the both outputs. Ultimately, there are four results:

\begin{itemize}
\item LL - result of LPF applied on both, columns and rows,
\item LH - result of LPF applied on columns and HPF on rows,
\item HL - result of HPF applied on columns and LPF on rows,
\item HH - result of HPF applied on both, columns and rows.
\end{itemize}  

\begin{figure}[h]
	\centering
	\includegraphics[width=\textwidth]{2D_DWT.JPG}
	\caption{2-D Discrete Wavelet transform on an image.}
	\label{fig:2D_DWT}
\end{figure}

Recall that the outcome of Low Pass Filter in the previous case was the main signal (without a noise). Thus, a 2-Dimensional equivalent is an approximation of an analysed image. The High Pass Filter captures high frequencies, then for an image the outcome are sudden changes in the image contrast. Now, lets focus on what exactly each result represent. First one, the LL is just an approximation of the initial image. Next, the LH shows abrupt changes in a horizontal direction, whereas the HL part presents similar issues but in a vertical direction. The HH shows sudden changes in a diagonal direction. In conclusion, the output of 2-D DWT gives us an approximation of the image and three parts with abrupt changes in different directions.

\section{Inverse Wavelet Transform}

Majority applications of Wavelet Transform assume that data is converted to wavelet domain and ten we can get the desired information. Although, we must return to the original domain for the information to be useful. The Inverse Wavelet Transform allows for such operation.

\begin{defn}
Inverse Discrete Wavelet Transform is defined as

\begin{equation}
y(t) = \sum_{j=1}^{K} \sum_{m=0}^{2^{K-j}-1} d^{(j,m)} 2^{-\frac{j}{2}} \psi\left(2^{-j}t - m\right) + \psi_0,
\end{equation}

where $y(t)$ is reconstruction of the input data and $d^{(j,m)}$ coefficients of wavelet transform. The $\psi_0$ is an average value of $y(t)$ over $t \in [0, 2^K-1]$. This parameter can be approximated by zero, without loss of generality.
\end{defn}

Thus, now we are able to transform data to a wavelet domain, process them simply and efficiently. As well as, we can back to the original domain to read the obtained results.

\chapter{Edge detection}

In this chapter let us focus on the main goal of the thesis, which is edge detection. What is an edge? It is a place where image brightness changes rapidly. There are various methods to identify such discontinuities. The most popular are gradient based (e.g. Canny, Prewitt, Sobel) and Laplacian based. However, there is also another method, which provides similar results and can be more efficient in terms of computation. This method is based on the 2-Dimensional Discrete Wavelet Transform described in section \ref{sec:2D_DWT}. \newline

What should be done? \textit{** Link to article Edge detection **}
\begin{enumerate}
\item Convert image to grey scale.
\item Apply 2-D DWT on an image.
\item Remove the LL part.
\item Denoise the LH, HL and HH coefficients.
\item Reconstruct the initial image.
\item Post-processing - modify contrast to emphasize obtained edges.
\end{enumerate}

\section{Implementation of an algorithm}

The algorithm is implemented in \texttt{Python} using library/package (?) \texttt{PyWavelets}. There are used also auxiliary libraries like: \texttt{numpy}, \texttt{matplotlib}, \texttt{PIL} and \texttt{scipy}. The \texttt{PyWavelets} package contains all features required to edge detection algorithm, i.e. 2D Forward and Inverse Discrete Wavelet Transform, build-in many wavelet functions and thresholding functionality (used to denoise coefficients).

Initially, a colour image must be simplified by conversion to grey scale. Edges are recognised as changes in brightness, so a single pixel should contains only information about a colour (black) intensity. Then we can apply the 2D DWT function. As a result, according to the description in section \ref{sec:2D_DWT}, we obtain four components: LL, LH, HL and HH. The last three contains information about rapid brightness changes. Thus, we can remove the LL component. Subsequently, remaining components can be denoised. It means, we can /pozbyc sie/ small coefficients by thresholding. There are two types of thresholding hard and soft. The hard one /nadaje/ zero value for coefficients below the set threshold, while the soft one works in the same way on the coefficients smaller than threshold, but additionally the coefficient bigger than the set threshold are ''shrinked'' towards zero /o jego wartosc/. More about setting the threshold is described in section \ref{sec:threshold}. We used the soft thresholding because then obtained edges are smoother.

The last main step in the algorithm is the reconstruction of the initial image, i.e. application an inverse 2-D DWT on the denoised coefficients. In the result, we obtain the edges of the image. At the end, we can do some post-processing to emphasize obtained lines. The background has grey colour and the edges are more white or black. Therefore, using simple mathematical calculations we can modify image to have black background and white edges. It is enough to get an absolute value, subtract 128 and then scale by multiplying 2 times.
\textit{** Add some more about values in an image array, here or in the previous paragraph **}

\section{Thresholding}
\label{sec:threshold}
 As it was mentioned before there is hard and soft thresholding. Lets denote $\lambda$ as threshold and $d$ as wavelet coefficient. The thresholdings are defined respectively
 
\begin{equation}
D^H(d|\lambda)=
\begin{cases}
	0,& \text{for } |d| \leq \lambda,\\
	d,& \text{for } |d| > \lambda.
\end{cases}
\end{equation}

\begin{equation}
D^S(d|\lambda)=
\begin{cases}
	0,& \text{for } |d| \leq \lambda,\\
	d-\lambda,& \text{for } d > \lambda, \\
	d+\lambda,& \text{for } d < -\lambda.
\end{cases}
\end{equation}

The key is how to set the $\lambda$ to denoise coefficients and do not loose any significant information.
\textit{** Some more about it and how I choose the threshold **}

\section{Example}

\textit{** Show how algorithm work on a simple square **}
\chapter{Application in medicine}

The usage of DWT in biomedical images is described in the article \cite{BiomedicalImages}, mainly classification problem.

\textit{ ** I am not sure if my result has any value but I put them below. Maybe I'll add some description that also edge detection can be useful in pre-processing before carrying out the classification algorithms **}

\begin{figure}[h]
	\centering
	\begin{subdiagram}{brain_MRI_tumour}{original}
	\end{subdiagram}
	
	\begin{subdiagram}{brain_MRI_tumour_haar_0}{detected edges}
	\end{subdiagram}
	
	\caption{A brain MRI - tumour.}
	\label{fig:brain1}
\end{figure}

\begin{figure}[h]
	\centering
	\begin{mainsubdiagrams2}{brain2}{original}{brain2_haar_0}{detected edges}
	\end{mainsubdiagrams2}
	
	\caption{A brain MRI - small tumour.}
	\label{fig:brain2}
\end{figure}

\begin{figure}[h]
	\centering
	\begin{mainsubdiagrams2}{lungs1}{original}{lungs1_haar_0}{detected edges}
	\end{mainsubdiagrams2}
	
	\caption{Lungs.}
	\label{fig:lungs}
\end{figure}

\begin{figure}[h]
	\centering
	\begin{mainsubdiagrams2}{broken_bone}{original}{broken_bone_haar_90}{detected edges}
	\end{mainsubdiagrams2}
	
	\caption{Broken bone - arm.}
	\label{fig:arm}
\end{figure}

\begin{figure}[h]
	\centering
	\begin{mainsubdiagrams2}{leg1}{original}{leg1_haar_95}{detected edges}
	\end{mainsubdiagrams2}
	
	\caption{Broken bone - leg.}
	\label{fig:leg}
\end{figure}
\chapter*{Summary}

In conclusion, the wavelet theory, described in the chapter~\ref{ch:theory}, has wide applications in science and engineering problems, including data mining tasks. It is showed that particular wavelet properties facilitate and improve data analysis. Also the wavelet transform algorithms have low computational complexity, which is a~huge advantage in case of big datasets operations.

The dimensionality reduction is one of the pre-processing tasks. As an example, the edge detection problem was considered. The chapter~\ref{ch:edge_detection} contains the wavelet-based algorithm for edges recognition and the results for various type of images with different parameters. 
%The algorithm was developed using $PyWavelets$ package in $Python$.
What characterize the Discrete Wavelet Transform, are the low and high frequency components. Thanks to those resolutions, we can filter only the desired informations. There are many different wavelets which can be used in wavelet transform, however Daubechies family is considered, because of their orthogonality and vanishing moments. It occurs that the simplest Haar wavelet provides the best results for the majority of images. Nevertheless, the other Daubechies wavelets are also useful, especially when the Haar fails. During the processing data, after transformation to the wavelet domain, we can also denoise coefficients using thresholding. It is showed that the soft thresholding leads to the smoother edges, hence it is more suitable for edge detection. We also considered different values of the threshold $\lambda$ for various images. Noisier pictures, or those with different sharpness require increase the threshold to obtain more accurate results.
Generally, changing the parameters, i.e. wavelet type and threshold $\lambda$, allows for the better fitting to the analysed image.

The chapter~\ref{ch:medical_application} presents an application of wavelet analysis in classification of the biomedical images. Wavelet transform helps with pre-processing like feature extraction. There are papers (\cite{MRI}, \cite{BiomedicalImages}), which show how wavelet-base algorithms can successfully recognize the disease, like cancer (brain, breast, lungs). We also introduced simple feature extraction, similar to the edge detection on the selected biomedical images.

Summarising, wavelets tend to be really great tool for data mining tasks, in pre-\allowbreak-processing and also as an kernel in the machine learning algorithms. We shown an example of usage -- wavelet-based edge detection, which provides good results and can be applied for various images thanks to the selection of parameters.
The second presented example, the wavelets usage in analysis of the biomedical images, shows that wavelet theory is already applied in real life problems. Moreover, remark that the presented examples are just a~few of many wavelets usages in data processing and currently the number of different papers with wavelet theory and applications is constantly growing.

\listoffigures
%\listoftables

%\bibliographystyle{unsrt} % w kolejności pojawienia
\bibliographystyle{abbrv} % w kolejności alfabetycznej
\bibliography{bibliography_KK}
\nocite{WaveletMethodsInDataMining}
\nocite{DataManing}
\nocite{WaveletSupportVectorMachine}
\nocite{WaveletNotes}
\nocite{EdgeDetection}
\nocite{BiomedicalImages}
\nocite{APrimerOnWavelets}

	
\end{document}