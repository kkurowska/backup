\chapter{Application in medicine}

The usage of DWT in biomedical images is described in the article \cite{BiomedicalImages}, mainly classification problem.

\textit{ ** I am not sure if my result has any value but I put them below. Maybe I'll add some description that also edge detection can be useful in pre-processing before carrying out the classification algorithms **}

\begin{figure}[h]
	\centering
	\begin{subdiagram}{brain_MRI_tumour}{original}
	\end{subdiagram}
	
	\begin{subdiagram}{brain_MRI_tumour_haar_0}{detected edges}
	\end{subdiagram}
	
	\caption{A brain MRI - tumour.}
	\label{fig:brain1}
\end{figure}

\begin{figure}[h]
	\centering
	\begin{mainsubdiagrams2}{brain2}{original}{brain2_haar_0}{detected edges}
	\end{mainsubdiagrams2}
	
	\caption{A brain MRI - small tumour.}
	\label{fig:brain2}
\end{figure}

\begin{figure}[h]
	\centering
	\begin{mainsubdiagrams2}{lungs1}{original}{lungs1_haar_0}{detected edges}
	\end{mainsubdiagrams2}
	
	\caption{Lungs.}
	\label{fig:lungs}
\end{figure}

\begin{figure}[h]
	\centering
	\begin{mainsubdiagrams2}{broken_bone}{original}{broken_bone_haar_90}{detected edges}
	\end{mainsubdiagrams2}
	
	\caption{Broken bone - arm.}
	\label{fig:arm}
\end{figure}

\begin{figure}[h]
	\centering
	\begin{mainsubdiagrams2}{leg1}{original}{leg1_haar_95}{detected edges}
	\end{mainsubdiagrams2}
	
	\caption{Broken bone - leg.}
	\label{fig:leg}
\end{figure}