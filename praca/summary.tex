\chapter*{Summary}

In conclusion, the wavelet theory, described in the chapter~\ref{ch:theory}, has wide applications in science and engineering problems, including data mining tasks. It is showed that particular wavelet properties facilitate and improve data analysis. Also the wavelet transform algorithms has low computational complexity, which is a~huge advantage in case of big datasets operations.

The dimensionality reduction is one of the pre-processing tasks. As an example, the edge detection problem was considered. The chapter~\ref{ch:edge_detection} contains the wavelet-based algorithm for edges recognition and the results for various type of images with different parameters. 
%The algorithm was developed using $PyWavelets$ package in $Python$.
What characterize the Discrete Wavelet Transform, are the low and high frequency components. Thanks to those resolutions, we can filter only the desired informations. There are many different wavelets which can be used in wavelet transform, however Daubechies family is considered, because of their orthogonality and vanishing moments. It occurs that the simplest Haar wavelet provides the best results for the majority of images. Nevertheless, the other Daubechies wavelets are also useful, especially when the Haar fails. During the processing data, after transformation to the wavelet domain, we can also denoise coefficients using thresholding. It is showed that the soft thresholding leads to the smoother edges, hence it is more suitable for edge detection. We also considered different values of the threshold $\lambda$ for various images. Noisier pictures, or those with different sharpness require increase the threshold to obtain more accurate results.
Generally, changing the parameters, i.e. wavelet type and threshold $\lambda$, allows for the better fitting to the analysed image.

The chapter~\ref{ch:medical_application} presents an application of wavelet analysis in classification of the biomedical images. Wavelet transform helps with pre-processing like feature extraction. There are papers, \cite{MRI} and \cite{BiomedicalImages}, which shows how wavelet-base algorithms can successfully recognize the disease, like cancer (brain, breast, lungs). We also introduced simple feature extraction, similar to the edge detection on the selected biomedical images.

Summarising, wavelets tend to be really great tool for data mining tasks, in pre-\allowbreak-processing and also as an kernel in the machine learning algorithms. We shown an example of usage -- wavelet-based edge detection, which provides good results and can be applied for various images thanks to the selection of parameters.
The second presented example, the wavelets usage in analysis of the biomedical images, shows that wavelet theory is already applied in real life problems. Moreover, remark that the presented examples are just a~few of many wavelets usages in data processing and currently the number of different papers with wavelet theory and applications is constantly growing.