\chapter*{Introduction}

\textit{ ** Few words about my thesis...
Generally what is a data mining, what are the wavelets, what are the applications in data mining and finally, what is the main purpose - edge detection. **}

%\textit{ ** VERY GOOD PARAGRAPH - JUST NEEDS TO BE WRITE WITH DIFFERENT WORDS **}
%	
%\textit{ Real world data sets are usually not directly suitable for performing Data
%Mining algorithms. They contain noise, missing values and may be inconsistent.
%In addition, real world data sets tend to be too large and high-dimensional.
%Wavelets provide a way to estimate the underlying function from the data. With
%the vanishing moment property of wavelets, we know that only some wavelet
%coefficients are significant in most cases. By retaining selective wavelet coefficients,
%wavelet transform could then be applied to denoising and dimensionality
%reduction. Moreover, since wavelet coefficients are generally decorrelated,
%we could transform the original data into wavelet domain and then carry out
%Data Mining tasks.}

Nowadays, people gather a huge amount of, often unstructured, data. High-\allowbreak-dimensionality, missing values or noisiness are only some of the problems that we have to face in data mining. Wavelets can be really helpful in this area, thanks to their properties. First of all, there exist efficient algorithms for wavelet transform. Low computational complexity is a big advantage in case of data mining analysis.

%At this point wavelets are usually the solution. Thanks to their properties we are able to underlying function from the data.
%Then Data Mining tasks are way easier to carry out. 
%But how it is done? Every wavelet consists with many different wavelet coefficients.
%We can specify, thanks to vanishing moment property, which of wavelet coefficients are significant in our case.
%Knowing this we can perform wavelet transform which makes our data more less noisy and dimensional reduced.
%It is possible because wavelet coefficients are generally decorrelated. \newline

Pre-processing:
\begin{itemize}
\item Denoising signals and images. The Wavelet noise reduction - transform data into wavelet domain, remove noise components (with lower frequency) and then back to the original domain.

\item Dimensionality reduction. The idea is to simplify data, by getting rid off the less relevant information. In a wavelet domain we retain only the largest coefficients. Then, after getting back to the original domain we obtain simplified data.
\end{itemize}


Machine learning processing:
\begin{itemize}
\item Clustering. Low frequency parts are correlated with regions of objects concentration and the high frequency parts correspond to the areas with sudden changes in the objects distribution. Thus, clustering can be conducted by recognising correlated components in the wavelet domain.

\item Classification. Wavelet-base algorithm in machine learning (2-D DWT) developed by Castelli 1996. Notable faster than the classic one.

\item Regression. Uses to predict future values based on historical data. Wavelet approach is the non-parametric regression. Similar to the dimensionality reduction case.

\item Distributed Data Mining. Thanks to the orthogonality, data mining tasks can be carried out on smaller subsets of data independently and than results can be combined.
\end{itemize}

