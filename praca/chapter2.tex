\chapter{Edge detection}

In this chapter let us focus on the main goal of the thesis, which is edge detection. What is an edge? It is a place where image brightness changes rapidly. There are various methods to identify such discontinuities. The most popular are gradient based (e.g. Canny, Prewitt, Sobel) and Laplacian based. However, there is also another method, which provides similar results and can be more efficient in terms of computation. This method is based on the 2-Dimensional Discrete Wavelet Transform described in section \ref{sec:2D_DWT}. \newline

What should be done? \textit{** Link to article Edge detection **}
\begin{enumerate}
\item Convert image to grey scale.
\item Apply 2-D DWT on an image.
\item Remove the LL part.
\item Denoise the LH, HL and HH coefficients.
\item Reconstruct the initial image.
\item Post-processing - modify contrast to emphasize obtained edges.
\end{enumerate}

\section{Implementation of an algorithm}

The algorithm is implemented in \texttt{Python} using library/package (?) \texttt{PyWavelets}. There are used also auxiliary libraries like: \texttt{numpy}, \texttt{matplotlib}, \texttt{PIL} and \texttt{scipy}. The \texttt{PyWavelets} package contains all features required to edge detection algorithm, i.e. 2D Forward and Inverse Discrete Wavelet Transform, build-in many wavelet functions and thresholding functionality (used to denoise coefficients).

Initially, a colour image must be simplified by conversion to grey scale. Edges are recognised as changes in brightness, so a single pixel should contains only information about a colour (black) intensity. Then we can apply the 2D DWT function. As a result, according to the description in section \ref{sec:2D_DWT}, we obtain four components: LL, LH, HL and HH. The last three contains information about rapid brightness changes. Thus, we can remove the LL component. Subsequently, remaining components can be denoised. It means, we can /pozbyc sie/ small coefficients by thresholding. There are two types of thresholding hard and soft. The hard one /nadaje/ zero value for coefficients below the set threshold, while the soft one works in the same way on the coefficients smaller than threshold, but additionally the coefficient bigger than the set threshold are ''shrinked'' towards zero /o jego wartosc/. More about setting the threshold is described in section \ref{sec:threshold}. We used the soft thresholding because then obtained edges are smoother.

The last main step in the algorithm is the reconstruction of the initial image, i.e. application an inverse 2-D DWT on the denoised coefficients. In the result, we obtain the edges of the image. At the end, we can do some post-processing to emphasize obtained lines. The background has grey colour and the edges are more white or black. Therefore, using simple mathematical calculations we can modify image to have black background and white edges. It is enough to get an absolute value, subtract 128 and then scale by multiplying 2 times.
\textit{** Add some more about values in an image array, here or in the previous paragraph **}

\section{Thresholding}
\label{sec:threshold}
 As it was mentioned before there is hard and soft thresholding. Lets denote $\lambda$ as threshold and $d$ as wavelet coefficient. The thresholdings are defined respectively
 
\begin{equation}
D^H(d|\lambda)=
\begin{cases}
	0,& \text{for } |d| \leq \lambda,\\
	d,& \text{for } |d| > \lambda.
\end{cases}
\end{equation}

\begin{equation}
D^S(d|\lambda)=
\begin{cases}
	0,& \text{for } |d| \leq \lambda,\\
	d-\lambda,& \text{for } d > \lambda, \\
	d+\lambda,& \text{for } d < -\lambda.
\end{cases}
\end{equation}

The key is how to set the $\lambda$ to denoise coefficients and do not loose any significant information.
\textit{** Some more about it and how I choose the threshold **}

\section{Example}

\textit{** Show how algorithm work on a simple square **}